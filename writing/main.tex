\documentclass[12pt, letterpaper]{article}
\usepackage[margin=0.8in]{geometry}
\usepackage[utf8]{inputenc}
\usepackage[numbers]{natbib}
\usepackage{graphicx}
\usepackage{xcolor}
\usepackage{wrapfig}
\usepackage{hyperref}

\title{Prediction of metastasis target site using ligand-receptor co-expression (human cancer cell lines in vitro and healthy organs in vivo)}
\author{Mark Keller}
\date{Spring 2021}

\begin{document}
\maketitle

\section{Introduction}
\subsection{Background}
According to the seed-and-soil hypothesis, metastasis occurs when circulating cancer cells distributed throughout the body happen upon microenvironments in secondary organs which are amenable to their survival \cite{paget_distribution_1889}.
For instance, circulating cells originating from a breast primary tumor may metastasize to the brain if the mutant cells have acquired an altered lipid metabolism and arrive at the brain, where a specialized metabolic state is required for survival \cite{jin_metastasis_2020}.
The seed-and-soil hypothesis offers an explanation for the observation that metastases are not distributed uniformly throughout all organs given a primary tumor type.


It is now clear that there are many factors which determine metastatic target sites, including signals in the secondary tissue microenvironment, properties of circulatory vessels (blood and lymphatic), and physical forces \cite{gupta_cancer_2006}.
However, it has remained difficult to comprehensively evaluate the predictive value of each of these variables.
The MetMap project is a first step toward testing the metastasis potential of hundreds of human cancer cell lines using mouse models and a pooled barcoding approach.
Using the proportion of barcodes mapped to each cancer cell line measured in each secondary tissue type by RNA sequencing, Jin et al. estimate the metastatic potential and penetrance of each cell line to target each secondary tissue type \cite{jin_metastasis_2020}.
Jin et al. also performed RNA-seq for each cancer cell line in vitro, facilitating comparisons of the pre-injection expression levels of each cell line.


The Tabula Muris Consortium has recently published single-cell RNA-seq measurements of 20 organs in healthy mice \cite{schaum_single-cell_2018}.
The Tabula Muris dataset benefits from standardized experimental and data processing methodologies, simplifying downstream cross-organ transcriptomic comparisons.


\subsection*{Acknowledgements}
I am grateful to the Miller Lab for welcoming me as a rotation student.


\bibliography{main}{}
\bibliographystyle{unsrt}

\end{document}
